% This file was converted to LaTeX by Writer2LaTeX ver. 1.0.2
% see http://writer2latex.sourceforge.net for more info
\documentclass[twoside,letterpaper]{article}
\usepackage{amsmath}
\usepackage{amssymb,amsfonts,textcomp}
\usepackage{array}
\usepackage[english]{babel}
\usepackage[small,bf]{caption}
\usepackage{color}
\usepackage[T1]{fontenc}
\usepackage{hhline}
\usepackage{hyperref}
\usepackage[latin1]{inputenc}
\usepackage{multirow}
\usepackage{supertabular}
\hypersetup{pdftex, colorlinks=true, linkcolor=blue, citecolor=blue, filecolor=blue, urlcolor=blue, pdftitle=SYSTEMS AND SOFTWARE REQUIREMENTS SPECIFICATION (SSRS) TEMPLATE, pdfauthor=Clinton Jeffery, pdfsubject=, pdfkeywords=}
\usepackage[pdftex]{graphicx}
% Outline numbering
\setcounter{secnumdepth}{5}
\renewcommand\thesection{\arabic{section}}
\renewcommand\thesubsection{\arabic{section}.\arabic{subsection}}
\renewcommand\thesubsubsection{\arabic{section}.\arabic{subsection}.\arabic{subsubsection}}
\renewcommand\theparagraph{\arabic{section}.\arabic{subsection}.\arabic{subsubsection}.\arabic{paragraph}}
\renewcommand\thesubparagraph{\arabic{section}.\arabic{subsection}.\arabic{subsubsection}.\arabic{paragraph}.\arabic{subparagraph}}
\makeatletter
\newcommand\arraybslash{\let\\\@arraycr}
\makeatother
% Page layout (geometry)
\setlength\voffset{-1in}
\setlength\hoffset{-1in}
\setlength\topmargin{0.5in}
\setlength\oddsidemargin{1in}
\setlength\evensidemargin{1in}
\setlength\textheight{8.278in}
\setlength\textwidth{6.5in}
\setlength\footskip{0.561in}
\setlength\headheight{0.5in}
\setlength\headsep{0.461in}
% Footnote rule
\setlength{\skip\footins}{0.0469in}
\renewcommand\footnoterule{\vspace*{-0.0071in}\setlength\leftskip{0pt}\setlength\rightskip{0pt plus 1fil}\noindent\textcolor{black}{\rule{0.25\columnwidth}{0.0071in}}\vspace*{0.0398in}}
% Pages styles
\makeatletter
\newcommand\ps@Standard{
  \renewcommand\@oddhead{\rmfamily University of Idaho CS Department Instructional Use\hfill \hfill NOT FOR RELEASE}
  \renewcommand\@evenhead{\@oddhead}
  \renewcommand\@oddfoot{\foreignlanguage{english}{\textcolor{black}{SSRS Page }}\foreignlanguage{english}{\textcolor{black}{\thepage{}}}}
  \renewcommand\@evenfoot{\@oddfoot}
  \renewcommand\thepage{\arabic{page}}
}

%% ENVIRONMENT TO SINGLE-SPACE ENUMERATION LINES
\newenvironment{my_enumerate}{
\begin{enumerate}
  \setlength{\itemsep}{1pt}
  \setlength{\parskip}{0pt}
  \setlength{\parsep}{0pt}}{\end{enumerate}
}

\newcommand\ps@Convertviii{
  \renewcommand\@oddhead{}
  \renewcommand\@evenhead{\@oddhead}
  \renewcommand\@oddfoot{}
  \renewcommand\@evenfoot{\@oddfoot}
  \renewcommand\thepage{\arabic{page}}
}
\newcommand\ps@Convertvii{
  \renewcommand\@oddhead{}
  \renewcommand\@evenhead{\@oddhead}
  \renewcommand\@oddfoot{}
  \renewcommand\@evenfoot{\@oddfoot}
  \renewcommand\thepage{\arabic{page}}
}
\newcommand\ps@Convertvi{
  \renewcommand\@oddhead{}
  \renewcommand\@evenhead{\@oddhead}
  \renewcommand\@oddfoot{}
  \renewcommand\@evenfoot{\@oddfoot}
  \renewcommand\thepage{\arabic{page}}
}
\newcommand\ps@Convertv{
  \renewcommand\@oddhead{}
  \renewcommand\@evenhead{\@oddhead}
  \renewcommand\@oddfoot{}
  \renewcommand\@evenfoot{\@oddfoot}
  \renewcommand\thepage{\arabic{page}}
}
\newcommand\ps@Convertiv{
  \renewcommand\@oddhead{}
  \renewcommand\@evenhead{\@oddhead}
  \renewcommand\@oddfoot{}
  \renewcommand\@evenfoot{\@oddfoot}
  \renewcommand\thepage{\arabic{page}}
}
\newcommand\ps@Convertii{
  \renewcommand\@oddhead{}
  \renewcommand\@evenhead{\@oddhead}
  \renewcommand\@oddfoot{}
  \renewcommand\@evenfoot{\@oddfoot}
  \renewcommand\thepage{\arabic{page}}
}
\newcommand\ps@FirstPage{
  \renewcommand\@oddhead{}
  \renewcommand\@evenhead{\@oddhead}
  \renewcommand\@oddfoot{}
  \renewcommand\@evenfoot{\@oddfoot}
  \renewcommand\thepage{\arabic{page}}
}
\makeatother
\pagestyle{Standard}
\setlength\tabcolsep{1mm}
\renewcommand\arraystretch{1.3}
% footnotes configuration
\makeatletter
\renewcommand\thefootnote{\arabic{footnote}}
\makeatother
\title{SYSTEMS AND SOFTWARE REQUIREMENTS SPECIFICATION (SSRS) TEMPLATE}
\author{Clinton Jeffery}
\date{2010-11-18T11:33:37.30}
\begin{document}




\clearpage
{\centering\bfseries
SYSTEMS AND SOFTWARE \ REQUIREMENTS SPECIFICATION (SSRS) FOR
\par}


\bigskip

{\centering\bfseries
Phunctional UML Editor
\\(pUML)
\par}


\bigskip


\bigskip


\bigskip

\begin{figure}
\centering
\includegraphics[width=3.5in]{uidahologo.jpg}
\end{figure}

\bigskip


\bigskip

{\centering\bfseries
Version 0.0
\par}

{\centering\bfseries
February 23, 2012
\par}


\bigskip


\bigskip

{\centering\bfseries
Prepared for:
\par}
{\centering\bfseries
Bruce Bolden
\par}
{\centering\bfseries
and
\par}
{\centering\bfseries
Dr. Clint Jeffery
\par}

\bigskip


\bigskip

{\centering\bfseries
Prepared by:
\par}

{\centering\bfseries
Josh Armstrong
\\Zach Curtis
\\Brian Bowles
\\Logan Evans
\\Jeremy Klas
\\Nathan Krussel
\\Maxine Major
\\Morgan Weir
\\David Wells
\\and
\\Xiaozhe Shen
\par}

{\centering\bfseries
University of Idaho
\par}

{\centering\bfseries
Moscow, ID \ 83844-1010
\par}

\clearpage{\centering\bfseries
pUML SSRS
\par}


\bigskip





{\centering\bfseries RECORD OF CHANGES \par}


\bigskip

\begin{flushleft}
\tablehead{}
\begin{supertabular}[c]{|m{0.75in}|m{1.0in}|m{1.5in}|m{0.25in}|m{2in}|c|}
\hline

\centering \bfseries Change
\centering \bfseries Number
&

\centering \bfseries Date
\par
&

\centering \bfseries Location of change\newline
\centering \bfseries(e.g., page or figure \#)
&

\centering \bfseries A\newline
\centering \bfseries M\newline
\centering \bfseries D  
&

\centering \bfseries Brief description\newline
\centering \bfseries of change
&
\bfseries Initials
\\\hline

\centering 1
& 12/10/2011
& ~
& \centering A
& Text material for section 3 of the SSRS document.
& LE

\\\hline
\centering 2
& 01/17/2012
& ~
& \centering A
& Added updated SSRS/SSDD pdf and TeX files
& MM

\\\hline
\centering 3
& 02/01/2012
& ~
& \centering M
& Updated SSRS and SSDD
& MM

\\\hline
\centering 4
& 02/15/2012
& ~
& \centering M
& Use cases expanded and descriptions added
& MM

\\\hline
\centering 4
& 02/23/2012
& ~
& \centering M
& Use case descriptions completed
& MM

\\\hline
\end{supertabular}
\end{flushleft}










{
\foreignlanguage{english}{*}\foreignlanguage{english}{\textbf{A}}\foreignlanguage{english}{
- ADDED
\ }\foreignlanguage{english}{\textbf{M}}\foreignlanguage{english}{ -
MODIFIED
\ }\foreignlanguage{english}{\textbf{D}}\foreignlanguage{english}{ -
DELETED}}

\clearpage{\centering\bfseries
\foreignlanguage{english}{\MakeUppercase{\ }}\foreignlanguage{english}{\MakeUppercase{pUML SSRS}}
\par}

{\centering\bfseries
TABLE OF CONTENTS
\par}


\bigskip

{\bfseries
Section\ \ Page}

\setcounter{tocdepth}{9}
\renewcommand\contentsname{}
\tableofcontents

\bigskip








\clearpage\clearpage\setcounter{page}{1}\pagestyle{Convertii}
\section[Introduction]{\rmfamily\bfseries
Introduction}

\subsection[IDENTIFICATION]{\rmfamily\bfseries
IDENTIFICATION}
{
The software system being considered for development is referred to as Phunctional UML Editor (pUML). \ The customer providing specifications
for the system is Professor Bruce Bolden at the University of Idaho. \ The ultimate
customer, or end-user, of the system will be University of Idaho Computer Science students and/or faculty. \ This is a new project effort, so the version under development is version 0.0.}

\subsection[PURPOSE]{\rmfamily\bfseries
PURPOSE}
{
The purpose of the system under development is to create and store UML diagrams.
\ While the system will be used by Computer Science students at the University of Idaho,
this document is intended to be read and understood by UICS software
designers and coders.}

\subsection[SCOPE]{\rmfamily\bfseries
SCOPE}
{
The pUML software was conceptualized as a Software Engineering class project, and was launched in September 2011 .  The pUML project is as of the date of this SSRS publication, incomplete, and has yet no aquirers, users, support agencies at this time. Upon completion, the pUML software will be available only for distribution to the University of Idaho Computer Science department, and will be supported by the development team. }

\subsection[DEFINITIONS, ACRONYMS, AND
ABBREVIATIONS]{\rmfamily\bfseries
DEFINITIONS, ACRONYMS, AND ABBREVIATIONS}

\bigskip

\begin{flushleft}
\tablehead{}
\begin{supertabular}{|m{1.3587599in}|m{5.00806in}|}
\hline
\centering \bfseries Term or
Acronym &
\centering\arraybslash \bfseries
Definition\\\hline
 Alpha test &
 Limited release(s) to selected,
outside testers\\\hline
 Beta test &
 Limited release(s) to cooperating
customers wanting early access to developing systems\\\hline
 Final test &
 aka, Acceptance test, release of
full functionality to customer for approval\\\hline
 DFD &
 Data Flow Diagram\\\hline
 SDD &
 Software Design Document, aka SDS,
Software Design Specification\\\hline
 SRS &
 Software Requirements
Specification\\\hline
 SSRS &
 System and Software Requirements
Specification\\\hline
~
 &
~
\\\hline

\end{supertabular}
\end{flushleft}
\subsection[REFERENCES]{\rmfamily\bfseries
REFERENCES}
{
There are no references to be cited for the pUML SSRS at this time.}

\subsection[OVERVIEW AND RESTRICTIONS]{\rmfamily\bfseries
OVERVIEW AND RESTRICTIONS}
{
This document is for limited release only to UI CS personnel working on
the project.}


\bigskip

{
Section 2 of this document describes the system under development from a
holistic point of view. \ Functions, characteristics, constraints,
assumptions, dependencies, and overall requirements are defined from
the system-level perspective.}


\bigskip

{
Section 3 of this document describes the specific requirements of the
system being developed. \ Interfaces, features, and specific
requirements are enumerated and described to a degree sufficient for a
knowledgeable designer or coder to begin crafting an architectural
solution to the proposed system.}


\bigskip

{
Section 4 provides the requirements traceability information for the
project. \ Each feature of the system is indexed by the SSRS
requirement number and linked to its SDD and test references.}


\bigskip

{
Sections 5 and up are appendices including original information and
communications used to create this document.}











\clearpage\section[OVERALL DESCRIPTION]{\rmfamily\bfseries
OVERALL DESCRIPTION}

\subsection[PRODUCT PERSPECTIVE]{\rmfamily\bfseries
PRODUCT PERSPECTIVE}
{
This product is independent of any other product, and as such, is self-contained.
}

\subsection[PRODUCT FUNCTIONS]{\rmfamily\bfseries
PRODUCT FUNCTIONS}
{
This product's primary function is to allow the user to create UML diagrams.  
The program will allow the user to create new diagrams, edit existing diagrams, and save them to access later.
}

\subsection[USER CHARACTERISTICS]{\rmfamily\bfseries
USER CHARACTERISTICS}
{
The intended user for the pUML software is a software engineer, with a need to organize
the parts of the software engineering project. This user is already familiar with
computers and generally has some experience in programming languages.
}

\subsection[CONSTRAINTS]{\rmfamily\bfseries
CONSTRAINTS}
{
Since the pUML project was developed as a class assignment,
further development of this project will halt if the University of Idaho
faculty overseeing this project decide that this project should to no longer continue.
}

\subsection[ASSUMPTIONS AND DEPENDENCIES]{\rmfamily\bfseries
ASSUMPTIONS AND DEPENDENCIES}
{
The requirements for the pUML software were dictated by University of Idaho Computer Science
Department faculty, and any further direction this project may take will depend on their decisions.  
Furthermore, should any decision be made, for example,  a new programming language must be utilized,
or different features are to be added/removed, this project could change.
}




\subsection[SYSTEM LEVEL (NON{}-FUNCTIONAL)
REQUIREMENTS]{\rmfamily\bfseries
SYSTEM LEVEL (NON-FUNCTIONAL) REQUIREMENTS}

\subsubsection[Site dependencies]{\rmfamily\bfseries
Site dependencies}
{
The pUML software has no dependencies on any external resources, such as internet access, etc..
Any modern operating system (2008+) should be sufficient to support the pUML software,
and since this software is cross-platform, there should be no complications.
}

\subsubsection[Safety, security and privacy requirements]{\rmfamily\bfseries
Safety, security and privacy requirements}
{
There are no safety, security or privacy requirements at this time.
}

\subsubsection[Performance requirements]{\rmfamily\bfseries
Performance requirements}
{
This software is to be supported on one terminal per install, and since there are no dependencies,
it may be installed on a theoretically infinite number of terminals. This software is not designed to
be remotely accessed, and as a result, one user per session is recommended as well.
The software has not been tested to determine efficient transaction times as of the date of this SSRS publication.
}

\subsubsection[System and software quality]{\rmfamily\bfseries System
and software quality}
{
The fully developed software should be available for use and reliably handle all requests 98 percent of the time.  
Undo and Redo options will be available to handle errors made on the part of the user.  
Earlier stored sessions are not a part of the software package at this time, but may be developed at a later release.
This software is not designed for any level of flexibility at this time, but a future release may permit
integration with other software environments. Testability has not been tested at this time.
}

\subsubsection[Packaging and delivery requirements]{\rmfamily\bfseries
Packaging and delivery requirements}
{
The executable system and all associated documentation (i.e., SSRS, SDD,
code listing, test plan (data and results), and user manual) will be
delivered to the customer on CD{\textquoteright}s and/or via email, as
specified by the customer at time of delivery. \ Although document
{\textquotedblleft}drops{\textquotedblright} will occur throughout the
system development process, the final, edited version of the above
documents will accompany the final, accepted version of the executable
system.}

\subsubsection[Personnel{}-related requirements]{\rmfamily\bfseries
Personnel-related requirements}
{
The system under development has no special personnel-related
characteristics. }

\subsubsection[Training{}-related requirements]{\rmfamily\bfseries
Training-related requirements}
{
No training materials or expectations are tied to this project other
than the limited help screens built into the software and the
accompanying user manual.}

\subsubsection[Logistics{}-related requirements]{\rmfamily\bfseries
Logistics-related requirements}
{
The pUML software is intended for use on University of Idaho Computer Science department computers as well as computer science students' personal computers including, at a minimum, operating systems Windows 7, Mac OSX, and Linux.
Any minimum hardware requirements lie outside the scope of the resources available,
and there are no software application dependencies at this time.
}

\subsubsection[Precedence and criticality of requirements]{\rmfamily\bfseries
Precedence and criticality of requirements}
{
All requirements have equal weight.}












\clearpage\section[SPECIFIC REQUIREMENTS]{\rmfamily\bfseries
SPECIFIC REQUIREMENTS}

\subsection[EXTERNAL INTERFACE REQUIREMENTS]{\rmfamily\bfseries
EXTERNAL INTERFACE REQUIREMENTS}

\subsubsection[Hardware Interfaces]{\rmfamily\bfseries
Hardware Interfaces}
{
\foreignlanguage{english}{\ }\foreignlanguage{english}
{
- Operating system and environment capable of running QT 4.2.
\\- Storage disk
\\  E.g., hard drive, SSD, or secondary flash. 15 MB of space on one of these storage disks will be required
to execute the pUML executable. Additional space will be required to save user generated projects.}}

\subsubsection[Software Interfaces]{\rmfamily\bfseries
Software Interfaces}
{ \foreignlanguage{english}{\ }\foreignlanguage{english}
{
- QT 4.2
\\- C++ compiler
\\  Note: Software interfaces are expected to change before the next release.
}}

\subsubsection[User Interfaces]{\rmfamily\bfseries
User Interfaces}
{
\foreignlanguage{english}{\ }\foreignlanguage{english}
{
- Monitor
\\  Since pUML is a graphical program, a sufficiently large and bright monitor is recommended.
\\- Keyboard
\\  The user will frequently need to fill in text fields.
\\- Mouse
\\  The majority of user interaction is through the mouse.}}

\subsubsection[Other Communication
Interfaces]{\rmfamily\bfseries
Other Communication Interfaces}
{
\foreignlanguage{english}{\ }\foreignlanguage{english}{No other interfaces are required. }}





\clearpage\setcounter{page}{1}\pagestyle{Convertv}

\subsection[SYSTEM FEATURES]{\rmfamily\bfseries
SYSTEM FEATURES}


\subsubsection{Use Cases and Descriptions}
{The following use cases represent three different stages at which options will be available to the user.}

\bigskip
\bigskip

% ************************************
% LAUNCH USE CASE & DESCRIPTIONS
% ************************************
\paragraph[\ Use Category]
{\ Launch} {These options will be available to the user upon the immediate launch of pUML.}

\begin{figure}[h]
\centering
\includegraphics[width=6.0in]{ucaseLaunch.jpg}
\caption{Launch Use Case}
\end{figure}

%%% CREATE NEW PROJECT USE CASE DESCRIPTION
\begin{flushleft}
\tablehead{}
\begin{tabular}{|m{2.0in} m{5.0in}|}
\hline
{\bfseries\emph{Use Case Name}}
& {\bfseries Create New Project}
\\\hline
\emph{Participating Actor}
& User
\\\hline
\emph{Flow of Events}
& \begin{my_enumerate}
  \item User opens File menu and selects ``Create New Project'' 
  \item A dialog appears asking the user to enter the project name and save location. 
    \item User clicks ``OK'' 
\item Program creates a new project for the user at the specified location, and opens the project in pUML.
\end{my_enumerate}
\\\hline
\emph{Entry Condition}
&
User opens File menu and selects ``Create New Project.''
\\\hline
\emph{Exit Condition}
& Project has been created and is open for modification in pUML.
\\\hline
\emph{Quality Requirements}
& Project name may not be the same as another project at the specified location. Project creation may only occur at program launch, and not while a project has been loaded into pUML.
\\\hline
\end{tabular}
\end{flushleft}
\bigskip

%%% OPEN EXISTING PROJECT USE CASE DESCRIPTION
\begin{flushleft}
\tablehead{}
\begin{tabular}{|m{2.0in} m{5.0in}|}
\hline
{\bfseries\emph{Use Case Name}}
& {\bfseries Open Existing Project}
\\\hline
\emph{Participating Actor}
& User
\\\hline
\emph{Flow of Events}
& \begin{my_enumerate}
\item User opens File menu and selects ``Open Project.''
\item User browses to desired directory and selects an existing pUML project file.
\item The program loads the specified project.
\end{my_enumerate}
\\\hline
\emph{Entry Condition}
& User opens File menu and selects ``Open Project.''
\\\hline
\emph{Exit Condition}
& Project has been loaded and is open for modification in pUML.
\\\hline
\emph{Quality Requirements}
& Project must be of pUML file type.
\\\hline
\end{tabular}
\end{flushleft}
\bigskip

%%% EXIT PROGRAM USE CASE DESCRIPTION
\begin{flushleft}
\tablehead{}
\begin{tabular}{|m{2.0in} m{5.0in}|}
\hline
{\bfseries\emph{Use Case Name}}
& {\bfseries Exit Program}
\\\hline
\emph{Participating Actor}
& User
\\\hline
\emph{Flow of Events}
& \begin{my_enumerate}
\item User clicks X.
\item If file is not saved, include ``Save File As'' use case.
\item Program Exits.
\end{my_enumerate}
\\\hline
\emph{Entry Condition}
&
User initiates File menu option ``Close Program'' or clicks X in the corner of the window.
\\\hline
\emph{Exit Condition}
& File is successfully saved and the program exits.
\\\hline
\emph{Quality Requirements}
& File must be saved prior to exiting, or changes must be discarded prior to exiting.
\\\hline
\end{tabular}
\end{flushleft}
\bigskip

\clearpage

% ************************************
% NEW PROJECT USE CASE & DESCRIPTIONS
% ************************************
\paragraph[\ Use Category]
{\ New Project} 
{The user has selected to start a new UML project.}

\begin{figure}[h]
\centering
\includegraphics[width=6.0in]{ucaseNewProj.jpg}
\caption{New Project Use Case}
\end{figure}

%%% CREATE NEW DIAGRAM USE CASE DESCRIPTION
\begin{flushleft}
\tablehead{}
\begin{tabular}{|m{2.0in} m{5.0in}|}
\hline
{\bfseries\emph{Use Case Name}}
& {\bfseries Create New Diagram }
\\\hline
\emph{Participating Actor}
& User
\\\hline
\emph{Flow of Events}
& \begin{my_enumerate}
\item The user selects new file from menu.
\item Program requests user select a diagram type
(includes ChooseDiagramType use case).
\item The program responds by creating a new file with a blank drawing canvas.
\end{my_enumerate}
\\\hline
\emph{Entry Condition}
& User selects New File from commands
\\\hline
\emph{Exit Condition}
& Program successfully opens new file
\\\hline
\emph{Quality Requirements}
& TBD
\\\hline
\end{tabular}
\end{flushleft}
\bigskip

%%% CHOOSE DIAGRAM TYPE USE CASE DESCRIPTION
\begin{flushleft}
\tablehead{}
\begin{tabular}{|m{2.0in} m{5.0in}|}
\hline
{\bfseries\emph{Use Case Name}}
& {\bfseries Choose Diagram Type}
\\\hline
\emph{Participating Actor}
& User 
\\\hline
\emph{Flow of Events}
& \begin{my_enumerate}
\item User selects a diagram type.
\item Program loads and displays only objects for the selected diagram type.
\end{my_enumerate}
\\\hline
\emph{Entry Condition}
& Included from New Diagram use case
\\\hline
\emph{Exit Condition}
& Included from New Diagram use case
\\\hline
\emph{Quality Requirements}
& Toolbar loads only diagram specific objects and connectors.
\\\hline
\end{tabular}
\end{flushleft}
\bigskip

\clearpage


% ************************************
% OPEN PROJECT USE CASE & DESCRIPTIONS
% ************************************
\paragraph[\ Use Category]
{\ Open Project} {These options will be available to the user after choosing to open an existing UML project.}

\begin{figure}[h]
\centering
\includegraphics[width=6.0in]{ucaseOpenProj.jpg}
\caption{Open Project Use Case}
\end{figure}

%%% OPEN EXISTING DIAGRAM USE CASE DESCRIPTION
\begin{flushleft}
\tablehead{}
\begin{tabular}{|m{2.0in} m{5.0in}|}
\hline
{\bfseries\emph{Use Case Name}}
& {\bfseries Open Existing Diagram}
\\\hline
\emph{Participating Actor}
& User
\\\hline
\emph{Flow of Events}
& \begin{my_enumerate}
\item User selects Open File from menu.
\item Program opens an explorer window.
\item User selects a file.
\item Program loads selected file into program
\end{my_enumerate}
\\\hline
\emph{Entry Condition}
& User selects Open File from menu.
\\\hline
\emph{Exit Condition}
& File is successfully opened.
\\\hline
\emph{Quality Requirements}
& User selected file must be able to be opened in pUML.
\\\hline
\end{tabular}
\end{flushleft}
\bigskip

\clearpage

% ************************************
% NEW DIAGRAM USE CASE & DESCRIPTIONS
% ************************************
\paragraph[\ Use Category]
{\ New Diagram} {These options will be available to the user upon choosing to create a new UML diagram.}

\begin{figure}[h]
\centering
\includegraphics[width=6.0in]{ucaseNewOpenDiagram.jpg}
\caption{New/Open Diagram Use Case}
\end{figure}

%%% SAVE DIAGRAM USE CASE DESCRIPTION
\begin{flushleft}
\tablehead{}
\begin{tabular}{|m{2.0in} m{5.0in}|}
\hline
{\bfseries\emph{Use Case Name}}
& {\bfseries Save File}
\\\hline
\emph{Participating Actor}
& User
\\\hline
\emph{Flow of Events}
& \begin{my_enumerate}
  \item The user opens the File menu and selects ``Save.''
  \item Program saves file under current file name 
\begin{my_enumerate}
\item If file has not been previously saved, program initiates Save As use case. 
\item If file has been previously saved, the program saves the file at its existing location under its existing name.
\end{my_enumerate}
\end{my_enumerate}
\\\hline
\emph{Entry Condition}
& User selects ``Save File'' from main menu or clicks Save button.
\\\hline
\emph{Exit Condition}
& File is successfully saved
\\\hline
\emph{Quality Requirements}
& TBD
\\\hline
\end{tabular}
\end{flushleft}
\bigskip


%%% SAVE DIAGRAM AS USE CASE DESCRIPTION
\begin{flushleft}
\tablehead{}
\begin{tabular}{|m{2.0in} m{5.0in}|}
\hline
{\bfseries\emph{Use Case Name}}
& {\bfseries Save Diagram As}
\\\hline
\emph{Participating Actor}
& User
\\\hline
\emph{Flow of Events}
& \begin{my_enumerate}
\item User opens File menu and selects ``Save As...''
\end{my_enumerate}
\ ~ or
\begin{my_enumerate}
\item User selects ``Save'' from File menu, but the program has not been previously saved.
\item Program displays a dialog box requesting the user to enter a diagram name.
\item User enters a diagram name and clicks ``Ok.''
\item Program saves diagram within existing project.
\end{my_enumerate}
\\\hline
\emph{Entry Condition}
& User clicks ``Save As'' or clicks ``Save'' with a previously unsaved diagram. 
\\\hline
\emph{Exit Condition}
& The diagram has been saved under the current working project.
\\\hline
\emph{Quality Requirements}
& Diagram name may not be the same as another diagram within the project. Diagram may not be saved outside of the project.
\\\hline
\end{tabular}
\end{flushleft}
\bigskip



%%% PRINT DIAGRAM USE CASE DESCRIPTION
\begin{flushleft}
\tablehead{}
\begin{tabular}{|m{2.0in} m{5.0in}|}
\hline
{\bfseries\emph{Use Case Name}}
& {\bfseries Print Diagram}
\\\hline
\emph{Participating Actor}
& User
\\\hline
\emph{Flow of Events}
& \begin{my_enumerate}
\item User initiates print. 
\item Program sends to printer
\end{my_enumerate}
\\\hline
\emph{Entry Condition}
& User initiates Print
\\\hline
\emph{Exit Condition}
& Diagram successfully sent to default printer
\\\hline
\emph{Quality Requirements}
& TBD
\\\hline
\end{tabular}
\end{flushleft}
\bigskip

%%% CLOSE DIAGRAM/TAB USE CASE DESCRIPTION
\begin{flushleft}
\tablehead{}
\begin{tabular}{|m{2.0in} m{5.0in}|}
\hline
{\bfseries\emph{Use Case Name}}
& {\bfseries Close Diagram}
\\\hline
\emph{Participating Actor}
& User
\\\hline
\emph{Flow of Events}
& \begin{my_enumerate}
\item User clicks the `X' on the diagram tab.
\item If the diagram is saved, the program closes the diagram tab.
\end{my_enumerate}
\ ~or
\begin{my_enumerate}
\item User clicks the `X' on the diagram tab.
\item If the program is not saved, the program issues a warning.
\end{my_enumerate}
\\\hline
\emph{Entry Condition}
& User clicks the ``X'' on the diagram tab.
\\\hline
\emph{Exit Condition}
& Diagram tab is closed.
\\\hline
\emph{Quality Requirements}
& Diagram must be saved for the tab to close. Else the user must either choose to save the diagram or discard changes.
\\\hline
\end{tabular}
\end{flushleft}
\bigskip


%%% UNDO USE CASE DESCRIPTION
\begin{flushleft}
\tablehead{}
\begin{tabular}{|m{2.0in} m{5.0in}|}
\hline
{\bfseries\emph{Use Case Name}}
& {\bfseries Undo}
\\\hline
\emph{Participating Actor}
& User
\\\hline
\emph{Flow of Events}
& \begin{my_enumerate}
\item User clicks ``Undo''
\item Program reverts to the state of the diagram prior to the last change made.
\end{my_enumerate}
\\\hline
\emph{Entry Condition}
& User clicks ``Undo.''
\\\hline
\emph{Exit Condition}
& The state of the diagram prior to the most recent change is now the active state.
\\\hline
\emph{Quality Requirements}
& Undo will not work if there are no changes made to the diagram during the current open state.
\\\hline
\end{tabular}
\end{flushleft}
\bigskip


%%% REDO USE CASE DESCRIPTION
\begin{flushleft}
\tablehead{}
\begin{tabular}{|m{2.0in} m{5.0in}|}
\hline
{\bfseries\emph{Use Case Name}}
& {\bfseries Redo}
\\\hline
\emph{Participating Actor}
& User
\\\hline
\emph{Flow of Events}
& \begin{my_enumerate}
\item User clicks ``Redo.''
\item Program advances to the state from which the user had previously clicked ``Undo.''
\end{my_enumerate}
\\\hline
\emph{Entry Condition}
&  User clicks ``Redo.''
\\\hline
\emph{Exit Condition}
& Current diagram loaded is now a state more current than the one from which the user had clicked ``Undo.''
\\\hline
\emph{Quality Requirements}
& Redo will not work if there are no changes made to the diagram during the current open state. The user must have previously clicked ``Undo'' in order for a  state to exist to redo.
\\\hline
\end{tabular}
\end{flushleft}
\bigskip



\clearpage

% ************************************
% EDIT OBJECT USE CASE & DESCRIPTIONS
% ************************************
\paragraph[\ Use Category]
{\ Edit Object} {These options will be available to the user with a diagram open for modification, who is choosing to take action on an object.}

\begin{figure}[h]
\centering
\includegraphics[width=6.0in]{ucaseEditObj.jpg}
\caption{Edit an Object Use Case}
\end{figure}

%%% PLACE NEW OBJECT USE CASE DESCRIPTION
\begin{flushleft}
\tablehead{}
\begin{tabular}{|m{2.0in} m{5.0in}|}
\hline
{\bfseries\emph{Use Case Name}}
& {\bfseries Place New Object}
\\\hline
\emph{Participating Actor}
& User
\\\hline
\emph{Flow of Events}
& \begin{my_enumerate}
\item User selects an object from toolbar.
\item User clicks the canvas.
\item Program places an object at that location on the canvas.
\end{my_enumerate}
\\\hline
\emph{Entry Condition}
& Toolbar is loaded with valid objects for diagram type.
\\\hline
\emph{Exit Condition}
& Object has been successfully placed on drawing canvas.
\\\hline
\emph{Quality Requirements}
& Object may not occupy the same space on the canvas as any other object.
\\\hline
\end{tabular}
\end{flushleft}
\bigskip

%%% EDIT OBJECT DESCRIPTION USE CASE DESCRIPTION
\begin{flushleft}
\tablehead{}
\begin{tabular}{|m{2.0in} m{5.0in}|}
\hline
{\bfseries\emph{Use Case Name}}
& {\bfseries Edit Object Description}
\\\hline
\emph{Participating Actor}
& User
\\\hline
\emph{Flow of Events}
& \begin{my_enumerate}
\item User selects an object by clicking on it.
\item User right-clicks on the object.
\item User selects ``Properties'' from the drop-down menu.
\item User types the desired description in the text box presented.
\item User clicks ``OK.''
\item Program displays new description.
\end{my_enumerate}
\\\hline
\emph{Entry Condition}
& User clicks on an object.
\\\hline
\emph{Exit Condition}
& Program is now showing all user-made changes.
\\\hline
\emph{Quality Requirements}
& There are limitations as to the number of characters a user may enter in the object description text field.
\\\hline
\end{tabular}
\end{flushleft}
\bigskip


%%% MOVE OBJECT USE CASE DESCRIPTION
\begin{flushleft}
\tablehead{}
\begin{tabular}{|m{2.0in} m{5.0in}|}
\hline
{\bfseries\emph{Use Case Name}}
& {\bfseries Move Object}
\\\hline
\emph{Participating Actor}
& User
\\\hline
\emph{Flow of Events}
& \begin{my_enumerate}
\item User clicks on object, selecting it.
\item User holds down the left mouse button, and moves the mouse to the new location of the object.
\item User releases the left mouse button.
\item Program draws the object at the new location, with all attached connectors. 
\end{my_enumerate}
\\\hline
\emph{Entry Condition}
& User selects an object with the left mouse button held down.
\\\hline
\emph{Exit Condition}
& Program has drawn the object at the new location.
\\\hline
\emph{Quality Requirements}
& User may not move an object to a location already occupied by another object. 
\\\hline
\end{tabular}
\end{flushleft}
\bigskip


\clearpage

% ************************************
% EDIT CONNECTOR CASE & DESCRIPTIONS
% ************************************
\paragraph[\ Use Category]
{\ Edit Connector} {These options will be available to the user with a diagram open for modification, who is choosing to take action regarding a connector.}

\begin{figure}[h]
\centering
\includegraphics[width=6.0in]{ucaseEditConn.jpg}
\caption{Edit a Connector Use Case}
\end{figure}

%%% PLACE NEW CONNECTOR USE CASE DESCRIPTION
\begin{flushleft}
\tablehead{}
\begin{tabular}{|m{2.0in} m{5.0in}|}
\hline {\bfseries\emph{Use Case Name}}
& {\bfseries Place New Connector}
\\\hline
\emph{Participating Actor}
& User
\\\hline
\emph{Flow of Events}
& \begin{my_enumerate}
\item User clicks on desired connector in toolbar.
\item User clicks on the object the connection is coming from, holding down the left mouse button.
\item User drags the mouse toward the object to be connected to.
\item User releases the mouse button.
\item Program draws the connection line between the first and second object.
\end{my_enumerate}
\\\hline
\emph{Entry Condition}
& User has selected a connector from the toolbar.
\\\hline
\emph{Exit Condition}
& The connector is successfully drawn between two objects. 
\\\hline
\emph{Quality Requirements}
& Only one connector may be placed between any two objects. Some objects may not be able to be connected with certain types of connectors.
\\\hline
\end{tabular}
\end{flushleft}
\bigskip


%%% EDIT CONNECTOR PROPERTIES USE CASE DESCRIPTION
\begin{flushleft}
\tablehead{}
\begin{tabular}{|m{2.0in} m{5.0in}|}
\hline {\bfseries\emph{Use Case Name}}
& {\bfseries Edit Connector Properties}
\\\hline
\emph{Participating Actor}
& User
\\\hline
\emph{Flow of Events}
& \begin{my_enumerate}
\item User selects a connector by clicking on it.
\item User right-clicks on the connector.
\item User selects ``Properties'' from the drop-down menu.
\item User edits propertes: 
	\begin{my_enumerate}
\item User types the desired description in the text box presented.
\item User selects desired arrowhead.
\end{my_enumerate}
\item User clicks ``OK.''
\item Program displays new description or arrowhead.
\end{my_enumerate}
\\\hline
\emph{Entry Condition}
& User right clicks on a connector.
\\\hline
\emph{Exit Condition}
& Program has updated the connector with the new properties.
\\\hline
\emph{Quality Requirements}
& Not all connectors will have the same property options.
\\\hline
\end{tabular}
\end{flushleft}
\bigskip


%%% DELETE CONNECTOR USE CASE DESCRIPTION
\begin{flushleft}
\tablehead{}
\begin{tabular}{|m{2.0in} m{5.0in}|}
\hline {\bfseries\emph{Use Case Name}}
& {\bfseries Delete Connector}
\\\hline
\emph{Participating Actor}
& User
\\\hline
\emph{Flow of Events}
& \begin{my_enumerate}
\item User right clicks on a connector.
\item User selects ``Delete'' from the drop-down menu.
\item Program deletes the connector.
\end{my_enumerate}
\\\hline
\emph{Entry Condition}
& User right clicks on the connector.
\\\hline
\emph{Exit Condition}
& Connector has been deleted.
\\\hline
\emph{Quality Requirements}
& TBD
\\\hline
\end{tabular}
\end{flushleft}
\bigskip



\clearpage






\clearpage


\subsubsection[System feature: [Project management tasks suite]{\rmfamily\bfseries System
feature: Project management tasks suite}

\paragraph[\ Introduction/Purpose of this feature]{\foreignlanguage{english}{\ }\foreignlanguage{english}{Introduction/Purpose of this feature}}
{
Several processes are required to manage projects. These include the saving of files, loading projects from files, and deleting projects.}


\paragraph[Input/Output sequence for this feature]{\rmfamily\bfseries
Input/Output sequence for this feature}
{
The user selects the ``File'' pull down menu at the top left corner of his or her screen. The options currently supported are ``New'', ``Open'', ``Revert'', ``Copy As'', ``Save'', ``Save As'', and ``Delete Project''. }

\subparagraph{New}
{
This creates a new project space. The user shall be presented with a dialog option to save the previous project. Once completing this dialog, the previous project will be closed and a new project space will be created. }

\subparagraph{Open}
{
This opens a previously saved project. The user shall be presented with a dialog option to save the previous project. Once completing this dialog, the previous project will be closed and the selected project will be opened.}

\subparagraph{Revert}
{
This disregards all unsaved changes made to the project and reverts to the saved version. A confirmation dialog shall be presented to the user before this action is completed.}

\subparagraph{Copy As}
{
This creates a new project space. First, the program will be saved. Then the user shall be presented with a dialog to choose a unique name for the new project space. Once completing this dialog, a copy of the previous project will be cloned into the new project space. The previous project shall be closed and the new project shall be opened.}

\subparagraph{Save}
{
This saves the current state of the project into an XML that can later be used to recreate the project state. If the current project does not have a name, the ``Save'' option will be an alias for the ``Save As'' option.}

\subparagraph{Save As}
{
This creates a new project space and then saves the project state into the new project. In contrast with ``Copy As'', this option does not affect the previous project. }

\subparagraph{Delete Project}
{
This deletes the current project and all files within the project folder. The user shall be presented with a confirmation dialog before this action is completed. Upon completion, a blank and unnamed project will be active in the project window. }









\clearpage\setcounter{page}{1}\pagestyle{Convertvi}
\section[REQUIREMENTS TRACEABILITY]{\rmfamily\bfseries
REQUIREMENTS TRACEABILITY}
{\itshape
This section shall contain traceability information from each system
requirement in this specification to the system (or subsystem, if
applicable) requirements it addresses. \ A tabular form is preferred,
but not mandatory.}


\bigskip

\begin{flushleft}
\tablehead{\hline
\multicolumn{1}{|m{0.9212598in}|}{\centering
\bfseries Feature Name} &
\centering \bfseries Req No. &
\centering \bfseries Requirement
Description &
\centering \bfseries Priority &
\centering \bfseries SDD &
\multicolumn{2}{m{1.2872598in}|}{\centering
\bfseries Alpha Release} &
\multicolumn{2}{m{1.3587599in}|}{\centering
\bfseries Beta Release} &
\multicolumn{2}{m{1.3795599in}|}{\centering
\bfseries Final Test}\\\hline
 &
 &
 &
 &
 &
\centering \bfseries Test Case(s) &
\centering \bfseries Test Res. &
\centering \bfseries Test Case(s) &
\centering \bfseries Test Res. &
\centering \bfseries Test Case(s) &
\bfseries Test
Res.\\\hhline{~~~~~------}}
\begin{supertabular}{m{0.9212598in}|m{0.42125985in}|m{1.9212599in}|m{0.39275986in}|m{0.7587598in}|m{0.6622598in}|m{0.5462598in}|m{0.6712598in}|m{0.6087598in}|m{0.6712598in}|m{0.6295598in}|}
\multicolumn{1}{|m{0.9212598in}|}{~Select Diagram Type
} &
\centering  1.1 &
~Selects the appropriate diagram type
 &
~M
 &
~N/A
 &
~N/A
 &
~N/A
 &
~N/A
 &
~N/A
 &
~N/A
 &

~N/A
\\\hline
\multicolumn{1}{|m{0.9212598in}|}{~Save function
} &
\centering  2.1 &
~Saves the Diagram to file
 &
~M
 &
~N/A
 &
~N/A
 &
~N/A
 &
~N/A
 &
~N/A
 &
~N/A
 &
~N/A
 
\\\hline
\multicolumn{1}{|m{0.9212598in}|}{~Draw function
} &
\centering  2.1 &
~Draws current objects
 &
~M
 &
~N/A
 &
~N/A
 &
~N/A
 &
~N/A
 &
~N/A
 &
~N/A
 &
~N/A

\\\hline
\multicolumn{1}{|m{0.9212598in}|}{~Open File
} &
\centering  2.1 &
~Opens previously saved file
 &
~M
 &
~N/A
 &
~N/A
 &
~N/A
 &
~N/A
 &
~N/A
 &
~N/A
 &
~N/A

\\\hline
\multicolumn{1}{|m{0.9212598in}|}{~New File
} &
\centering  2.1 &
~Creates New File
 &
~M
 &
~N/A
 &
~N/A
 &
~N/A
 &
~N/A
 &
~N/A
 &
~N/A
 &
~N/A

\\\hline

\multicolumn{1}{|m{0.9212598in}|}{~SSRS and SSDD
} &
\centering  2.1 &
~Too much work.
 &
~M
 &
~N/A
 &
~N/A
 &
~N/A
 &
~N/A
 &
~N/A
 &
~N/A
\\\hline

\end{supertabular}
\end{flushleft}
{
Priorities are: \textbf{M}andatory, \textbf{L}ow, \textbf{H}igh}

{
SDD link is version and page number or function name.}

{
Test cases and results are file names and \textbf{P}ass/\textbf{F}ail or
\% passing.}


\end{document}
