% This file was converted to LaTeX by Writer2LaTeX ver. 1.0.2
% see http://writer2latex.sourceforge.net for more info
\documentclass[twoside,letterpaper]{article}
\usepackage[latin1]{inputenc}
\usepackage[T1]{fontenc}
\usepackage[english]{babel}
\usepackage{amsmath}
\usepackage{amssymb,amsfonts,textcomp}
\usepackage{color}
\usepackage{array}
\usepackage{supertabular}
\usepackage{hhline}
\usepackage{hyperref}
\usepackage{verbatim}
\hypersetup{pdftex, colorlinks=true, linkcolor=blue, citecolor=blue, filecolor=blue, urlcolor=blue, pdftitle=SYSTEM AND SOFTWARE ARCHITECTURAL AND DETAILED DESIGN DESCRIPTI, pdfauthor=Clinton Jeffery, pdfsubject=, pdfkeywords=}
\usepackage[pdftex]{graphicx}
% Outline numbering
\setcounter{secnumdepth}{5}
\renewcommand\thesection{\arabic{section}}
\renewcommand\thesubsection{\arabic{section}.\arabic{subsection}}
\renewcommand\thesubsubsection{\arabic{section}.\arabic{subsection}.\arabic{subsubsection}}
\renewcommand\theparagraph{\arabic{section}.\arabic{subsection}.\arabic{subsubsection}.\arabic{paragraph}}
\renewcommand\thesubparagraph{\arabic{section}.\arabic{subsection}.\arabic{subsubsection}.\arabic{paragraph}.\arabic{subparagraph}}
\makeatletter
\newcommand\arraybslash{\let\\\@arraycr}
\makeatother
% List styles
\newcommand\liststyleWWviiiNumii{%
\renewcommand\theenumi{\arabic{enumi}}
\renewcommand\theenumii{\arabic{enumii}}
\renewcommand\theenumiii{\arabic{enumiii}}
\renewcommand\theenumiv{\arabic{enumiv}}
\renewcommand\labelenumi{\theenumi)}
\renewcommand\labelenumii{\theenumii.}
\renewcommand\labelenumiii{\theenumiii.}
\renewcommand\labelenumiv{\theenumiv.}
}
% Page layout (geometry)
\setlength\voffset{-1in}
\setlength\hoffset{-1in}
\setlength\topmargin{0.5in}
\setlength\oddsidemargin{1in}
\setlength\evensidemargin{1in}
\setlength\textheight{8.278in}
\setlength\textwidth{6.5in}
\setlength\footskip{0.561in}
\setlength\headheight{0.5in}
\setlength\headsep{0.461in}
% Footnote rule
\setlength{\skip\footins}{0.0469in}
\renewcommand\footnoterule{\vspace*{-0.0071in}\setlength\leftskip{0pt}\setlength\rightskip{0pt plus 1fil}\noindent\textcolor{black}{\rule{0.25\columnwidth}{0.0071in}}\vspace*{0.0398in}}
% Pages styles
\makeatletter
\newcommand\ps@Standard{
  \renewcommand\@oddhead{}
  \renewcommand\@evenhead{\@oddhead}
  \renewcommand\@oddfoot{{\textcolor{black}{\hfill Test Plan Page }}{{\textcolor{black}{\thepage{}}}}}
  \renewcommand\@evenfoot{\@oddfoot}
  \renewcommand\thepage{\arabic{page}}
}
\newcommand\ps@Convertix{
  \renewcommand\@oddhead{}
  \renewcommand\@evenhead{\@oddhead}
  \renewcommand\@oddfoot{}
  \renewcommand\@evenfoot{\@oddfoot}
  \renewcommand\thepage{\arabic{page}}
}
\newcommand\ps@Convertviii{
  \renewcommand\@oddhead{}
  \renewcommand\@evenhead{\@oddhead}
  \renewcommand\@oddfoot{}
  \renewcommand\@evenfoot{\@oddfoot}
  \renewcommand\thepage{\arabic{page}}
}
\newcommand\ps@Convertvii{
  \renewcommand\@oddhead{}
  \renewcommand\@evenhead{\@oddhead}
  \renewcommand\@oddfoot{}
  \renewcommand\@evenfoot{\@oddfoot}
  \renewcommand\thepage{\arabic{page}}
}
\newcommand\ps@Convertvi{
  \renewcommand\@oddhead{}
  \renewcommand\@evenhead{\@oddhead}
  \renewcommand\@oddfoot{}
  \renewcommand\@evenfoot{\@oddfoot}
  \renewcommand\thepage{\arabic{page}}
}
\newcommand\ps@Convertiv{
  \renewcommand\@oddhead{}
  \renewcommand\@evenhead{\@oddhead}
  \renewcommand\@oddfoot{}
  \renewcommand\@evenfoot{\@oddfoot}
  \renewcommand\thepage{\arabic{page}}
}
\newcommand\ps@FirstPage{
  \renewcommand\@oddhead{}
  \renewcommand\@evenhead{\@oddhead}
  \renewcommand\@oddfoot{{\textcolor{black}{\hfill TP Page }}{{\textcolor{black}{\thepage{}}}}}
  \renewcommand\@evenfoot{\@oddfoot}
  \renewcommand\thepage{\arabic{page}}
}
\makeatother
\pagestyle{Standard}
\setlength\tabcolsep{1mm}
\renewcommand\arraystretch{1.3}
\title{TEST PLAN}
\author{Clinton Jeffery}
\date{2011-02-09}
\begin{document}
\clearpage\setcounter{page}{1}\pagestyle{Standard}
\thispagestyle{FirstPage}

{\centering\bfseries
TEST PLAN (TP)
\par}

{\centering\bfseries
FOR
\par}


\bigskip

{\centering\bfseries
Phunctional UML Editor
\\(pUML)
\par}


\bigskip


\bigskip


\bigskip

{\centering \par}

\begin{figure}
\centering
\includegraphics[width=3.5in]{uidahologo.jpg}
\end{figure}

\bigskip


\bigskip


\bigskip


\bigskip


{\centering\bfseries
Version 2.0
\par}

{\centering\bfseries
April 9, 2012
\par}


\bigskip


\bigskip

{\centering\bfseries
Prepared for:
\par}

{\centering\bfseries
Dr. Clint Jeffery
\par}


\bigskip



{\centering\bfseries
Prepared by:
\par}

{\centering\bfseries
Josh Armstrong
\\Zach Curtis
\\Brian Bowles
\\Logan Evans
\\Jeremy Klas
\\Nathan Krussel
\\Maxine Major
\\Morgan Weir
\\David Wells
\par}


\bigskip

{\centering\bfseries
University of Idaho
\par}

{\centering\bfseries
Moscow, ID \ 83844-1010
\par}

\pagebreak


{\centering\bfseries
CS384 TPD
\par}



{\centering\bfseries
RECORD OF CHANGES (Change History)
\par}

\begin{flushleft}
\tablehead{}
\begin{supertabular}{|m{0.5462598in}|m{0.6712598in}|m{1.4212599in}|m{0.23375985in}|m{1.7962599in}|m{0.7337598in}|m{0.6295598in}|}
\hline
~

\centering {\bfseries Change}\par
\centering {\bfseries Number}\par
&

\centering \bfseries Date completed
&

\centering {\bfseries Location of
change }\par
\centering \bfseries (e.g., page or
figure \#) &

\centering {\bfseries A}\par
\centering \bfseries M\newline
D
&

\centering {\bfseries Brief description }\par
\centering \bfseries of change 
&

\centering \bfseries Approved by
(initials) 
&

\centering {\bfseries Date }\par
\centering\arraybslash\bfseries
approved\\
 &
 &
 &
 &
 &
 &
\\\hline
01 & ~3/27/2012 & Test Plan & A & Document Created & MM & 3/27/2012
\\\hline
02 & 4/9/2012 & Test Plan & A & Test cases added & MM &4/9/2012
\\\hline
~ & ~ & ~ & ~ & ~ & ~ & ~
\\\hline
~ & ~ & ~ & ~ & ~ & ~ & ~
\\\hline
~ & ~ & ~ & ~ & ~ & ~ & ~
\\\hline
~ & ~ & ~ & ~ & ~ & ~ & ~
\\\hline
~ & ~ & ~ & ~ & ~ & ~ & ~
\\\hline
~ & ~ & ~ & ~ & ~ & ~ & ~
\\\hline
~ & ~ & ~ & ~ & ~ & ~ & ~
\\\hline
~ & ~ & ~ & ~ & ~ & ~ & ~
\\\hline
~ & ~ & ~ & ~ & ~ & ~ & ~
\\\hline
~ & ~ & ~ & ~ & ~ & ~ & ~
\\\hline
~ & ~ & ~ & ~ & ~ & ~ & ~
\\\hline
~ & ~ & ~ & ~ & ~ & ~ & ~
\\\hline
~ & ~ & ~ & ~ & ~ & ~ & ~
\\\hline
~ & ~ & ~ & ~ & ~ & ~ & ~
\\\hline
~ & ~ & ~ & ~ & ~ & ~ & ~
\\\hline
~ & ~ & ~ & ~ & ~ & ~ & ~
\\\hline
~ & ~ & ~ & ~ & ~ & ~ & ~
\\\hline
~ & ~ & ~ & ~ & ~ & ~ & ~
\\\hline
~ & ~ & ~ & ~ & ~ & ~ & ~
\\\hline
\end{supertabular}
\end{flushleft}
{
A - ADDED \ M - MODIFIED \ D -- DELETED}

\clearpage

{\centering\bfseries
pUML
\par}

{\centering\bfseries
TABLE OF CONTENTS
\par}

{\bfseries
Section\ \ Page}

\setcounter{tocdepth}{9}
\renewcommand\contentsname{}
\tableofcontents

\bigskip

\bigskip

\setcounter{page}{1}\pagestyle{Convertiv}

\clearpage

\section[IDENTIFIER]{\bfseries TEST PLAN IDENTIFIER}
\begin{comment}
\itshape Some type of unique company generated number to 
identify this test plan, its level and the level of software 
that it is related to. Preferably the test plan level will be
the same as the related software level. The number may also 
identify whether the test plan is a Master plan, a Level plan,
an integration plan or whichever plan level it represents. This 
is to assist in coordinating software and testware versions 
within configuration management. 
\end{comment}

{\bfseries This is test plan No. 007. } 

\bigskip

The pUML project team is not associated with an established company at this time, and this will be the only unique numerical identification number for any of our products.

\section[REFERENCES]{\bfseries REFERENCES}
\begin{comment}
{\itshape List all documents that support this test plan. Refer to the actual version/release number of the document as stored in the configuration management system. Do not duplicate the text from other documents as this will reduce the viability of this document and increase the maintenance effort.}
\end{comment}

\begin{itemize}
\item   Systems and Software Requirements Specification (SSRS) ver. 0.0
\item   System and Software Design Description (SSDD) ver. 0.0
\end{itemize}

\section[INTRODUCTION]{\bfseries INTRODUCTION}
\begin{comment}
{\itshape State the purpose of the Plan, possibly identifying the level of the plan (master etc.). This is essentially the executive summary part of the plan. }
\end{comment}

The purpose of this Test Plan is to ensure the integrity of the pUML software through a well-defined series of tests. The testing outlined in this plan will be applied to each component of the pUML software. Errors will be well-documented and each test will require a follow-up so all changes or recommended improvements to both functionality and features may be applied to this software.

\bigskip

The testing required will include manual testing, unit testing, and a combination of both and/or other specialized tests as necessary for each test item. 
\ results for each test item will be logged.
\ action taken on each test item will be logged
\ The purpose for this test is to ensure that all errors are found and handled. 

\section[TEST ITEMS]{\bfseries TEST ITEMS}
\begin{comment}
{\itshape These are things you intend to test within the scope of this test plan. Essentially, something you will test, a list of what is to be tested. This can be developed from the software application inventories as well as other sources of documentation and information.}
\end{comment}

Items to be tested include:
\begin{itemize}
\item   Installers
\item   Multiplatform portability
\item   Legality
\item   Functions and parameters
\item   Excessive code complexity
\item   Large program components, i.e. window and canvas class
\item   
\end{itemize}

\section[SOFTWARE RISK ISSUES]{\bfseries SOFTWARE RISK ISSUES} \begin{comment}
{\itshape Identify what software is to be tested and what the critical areas are, such as:}
\end{comment}

Software risk areas include extremely complex functions and modifications made on components with a history of failure. These functions and components are itemized as follows: 
\begin{itemize}
\item   QPaint. This introduces a complex hierarchy of classes and functions. To ensure this is adequately functional, manual testing will be required.
\item   Compilation issues.
\item   Restoring object classes. This tends to be fragile.
\end{itemize}


\section[FEATURES TO BE TESTED]{\bfseries FEATURES TO BE TESTED} \begin{comment}
{\itshape This is a listing of what is to be tested from the USERS viewpoint of what the system does. This is not a technical description of the software, but a USERS view of the functions.}
\end{comment}

Features to be tested include all objects, connectors, and associated functionality, Open/Save/Restore functionality, and ensuring that all diagram types load properly.

\subsection[]{\bfseries Black Box Testing} 

All black box tests will be conducted by students in the Computer Science department. 

\begin{flushleft}
\tablehead{}
\begin{tabular}{|m{1.5in} m{3.0in} m{2.0in}|}
\hline
{\bfseries\centering Name of Test}
& {\bfseries\centering Tests}
& {\bfseries\centering Fulfils SSRS Req.}
\\\hline
testinstall & pUML installer successfully installs and uninstalls & ~
\\\hline
testlaunch & pUML launches correctly & ~
\\\hline
testsave & All save functions & ~
\\\hline
testopen & All open functions & ~	
\\\hline
testmainwindow & All main window options & ~
\\\hline
testobjects & All object behavior & ~
\\\hline
testconnectors & All connector behavior & ~
\\\hline

\end{tabular}
\end{flushleft}
\bigskip

\subsection[]{\bfseries Integration Testing} 

The integration test will be conducted as a unit test, testing all signals and slots and connections within pUML.

\begin{flushleft}
\tablehead{}
\begin{tabular}{|m{1.5in} m{3.0in} m{2.0in}|}
\hline
{\bfseries Name of Test}
& {\bfseries Tests}
& {\bfseries Fulfils SSRS Req.}
\\\hline
connectCanvasWithDocument & connectCanvasWithDocumentTest & ~
\\\hline
\end{tabular}
\end{flushleft}
\bigskip


\subsection[]{\bfseries Unit Testing} 

The unit tests will be conducted as part of a QT unit test project. This project will test all units.

\begin{flushleft}
\tablehead{}
\begin{tabular}{|m{1.5in} m{3.0in} m{2.0in}|}
\hline
{\bfseries Name of Test}
& {\bfseries Tests}
& {\bfseries Fulfils SSRS Req.}
\\\hline
pUMLUnitTest & all pUML functions & ~
\\\hline
\end{tabular}
\end{flushleft}
\bigskip








\section[FEATURES NOT TO BE TESTED]{\bfseries
FEATURES NOT TO BE TESTED}
\begin{comment}
{\itshape This is a listing of what is NOT to be tested from both the Users viewpoint of what the system does and a configuration management/version control view. This is not a technical description of the software, but a USERS view of the functions.}
\end{comment}

\begin{itemize}
\item We will not test invalid Save file names. QT already has implemented measures to prevent invalid file names.
\item We will not test what the right-click function does in areas we have not assigned right-click features.
\end{itemize}



\section[APPROACH]{\bfseries APPROACH}
\begin{comment}
{\itshape This is your overall test strategy for this test plan; it should be appropriate to the level of the plan (master, acceptance, etc.) and should be in agreement with all higher and lower levels of plans. Overall rules and processes should be identified. 

\begin{itemize}
\item Are any special tools to be used? What are they?
\item What metrics will be collected for this test?
\item How many configurations/platforms are to be tested?
\item How will elements in the design deemed "untestable" be processed?
\end{itemize}
}
\end{comment}

We will be utilizing QT's built in unit testing features by designing a project to test all units. 
Black box testing will be conducted by Computer Science students per text files outlining all steps to complete the test. Failed tests will be logged as issues in Google code and will be resolved manually.


\section[ITEM PASS/FAIL CRITERIA]
{\bfseries ITEM PASS/FAIL CRITERIA}
\begin{comment}
{\itshape What are the Completion criteria for this plan? This is a critical aspect of any test plan and should be appropriate to the level of the plan.}
\end{comment}

If any test fails in any aspect, then entire test will have been assumed to have failed. A test will not be considered to pass until all steps of that test have passed successfully. If a test continues to fail and seems unresolvable given time and resources, the test may be rewritten to ensure a satisfactory pass grade may occur.


\section[SUSPENSION CRITERIA]
{\bfseries SUSPENSION CRITERIA AND RESUMPTION 
\\REQUIREMENTS}
\begin{comment}
{\itshape If the number or type of defects reaches a point where the follow on testing has no value, it makes no sense to continue the test; you are just wasting resources.
Specify what constitutes stoppage for a test or series of tests and what is the acceptable level of defects that will allow the testing to proceed past the defects. }
\end{comment}

Features and desired functionality may be suspended if it is assumed that it cannot be reasonably functional within the remaining time left to complete this project. 
Features to be suspended at this time will not be resumed.

\section[TEST DELIVERABLES]
{\bfseries TEST DELIVERABLES}
\begin{comment}
{\itshape What is to be delivered as part of this plan?
\begin{itemize}
\item Test plan document
\item Test cases
\item Relevant error logs or problem reports
\end{itemize}
One thing that is not a test deliverable is the software itself that
is listed under test items and is delivered by development.
}
\end{comment}

\begin{itemize}
\item Test plan document
\item All black box test cases with extraneous observations
\item The integration test runs 
\item All unit test runs 
\item Relevant error logs and problem reports
\end{itemize}


\section[REMAINING TEST TASKS]{\bfseries REMAINING TEST TASKS}
\begin{comment}
{\itshape If this is a multi-phase process or if the application is to be released in increments there may be parts of the application that this plan does not address. These areas need to be identified to avoid any confusion should defects be reported back on those future functions. This will also allow the users and testers to avoid incomplete functions and prevent waste of resources chasing non-defects.}
\end{comment}

There are no remaining test tasks at this time, since further testing will not occur at the end of this semester.

\section[ENVIRONMENTAL NEEDS]{\bfseries ENVIRONMENTAL NEEDS}
\begin{comment}
{\itshape Are there any special requirements for this test plan, such as:
\begin{itemize}
\item Special hardware such as simulators, static generators etc. 
\item How will test data be provided. Are there special collection requirements or specific ranges of data that must be provided? 
\end{itemize}
}
\end{comment}

There are no environmental needs associated with the pUML project.


\section[STAFFING AND TRAINING NEEDS]
{\bfseries STAFFING AND TRAINING NEEDS}
\begin{comment}
{\itshape Training on the application/system.
Training for any test tools to be used. }
\end{comment}

pUML users are assumed to be familiar with UML and have a need to perform UML diagramming. As such, basic training will not be necessary.  A user guide will be provided with the pUML software to address any additional concerns. 


\section[RESPONSIBILITIES]{\bfseries RESPONSIBILITIES}
{\itshape
Who is in charge?
This issue includes all areas of the plan. Here are some examples:
\begin{itemize}
\item Selecting features to be tested and not tested.
\item Ensuring all required elements are in place for testing. 
\end{itemize} }

Specific testing responsibilities are still TBD.


\section[SCHEDULE]{\bfseries SCHEDULE}
\begin{comment}
{\itshape Should be based on realistic and validated estimates. If the estimates for the development of the application are inaccurate, the entire project plan will slip and the testing is part of the overall project plan.}
\end{comment}

All black box tests, unit tests, and the integration test will be conducted on a weekly basis, as soon as test creation has been completed.

\section[PLANNING RISKS AND CONTINGENCIES]{\bfseries PLANNING RISKS AND CONTINGENCIES}
\begin{comment}
{\itshape What are the overall risks to the project with an emphasis on the testing process? Specify what will be done for various risk events.}
\end{comment}

[Insert text here.]

\section[APPROVALS]{\bfseries APPROVALS}
\begin{comment}
{\itshape Who can approve the process as complete and allow the project to proceed to the next level (depending on the level of the plan)? }
\end{comment}

Dr. Clint Jeffery is the only authorized individual to approve this project as complete.


				\begin{comment}
\section[GLOSSARY]
{\bfseries GLOSSARY}
			\end{comment}
\begin{comment}
{\itshape Used to define terms and acronyms used in the document, and testing in general, to eliminate confusion and promote consistent communications. }
\end{comment}
			



			\begin{comment}


\clearpage\setcounter{page}{1}\pagestyle{Convertviii}
\section[APPENDIX A. \ [insert name here{]}]
{\bfseries APPENDIX A.
\ [insert name here]
}
{\itshape Include copies of test examples, etc. supplied or
derived from the customer. \ Appendices are labeled A, B, {\dots}n. \ \ Reference each appendix as appropriate in the text of the document. }

\ [ insert appendix A here ]

\clearpage\setcounter{page}{1}\pagestyle{Convertix}

				\end{comment}



\bigskip
\end{document}

